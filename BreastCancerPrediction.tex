\documentclass{article}
\usepackage{blindtext}
\usepackage{url}
\usepackage{graphicx}

\title{Breast Cancer Prediction}
\author{Chiara Coscarelli\\[1ex]matr. 0512113869}

\begin{document}

\maketitle

\newpage
\tableofcontents
\newpage
\section{Introduzione}

Il cancro al seno (BC) è uno dei tumori più comuni tra le donne in tutto il mondo e, secondo le statistiche globali, rappresenta la maggior parte dei nuovi casi di cancro e dei decessi correlati al cancro, rendendolo un problema di salute pubblica significativo nella società odierna. La diagnosi precoce del BC può migliorare significativamente la prognosi e le possibilità di sopravvivenza, poiché può promuovere un trattamento clinico tempestivo per i pazienti. Una classificazione più accurata dei tumori benigni può evitare che i pazienti si sottopongano a trattamenti non necessari. Pertanto, la diagnosi corretta di BC e la classificazione dei pazienti in gruppi maligni o benigni è oggetto di molte ricerche.  <ADDlink al GitHub>


\subsection{Obiettivi}
Questo progetto di Fondamenti di Intelligenza Artificiale, il mio primo approccio in questo ambito, mira a osservare quali caratteristiche sono più utili nel predire il cancro maligno o benigno e a vedere le tendenze generali che potrebbero aiutarci nella selezione del modello. L’obiettivo è classificare se il cancro al seno è benigno o maligno. Ho scelto questa tematica perché relativamente semplice, e quindi in grado di farmi apprendere il più possibile le basi del Machine Learning.
Gli obiettivi principali includono:
\begin{itemize}
    \item L'analisi approfondita dei dati estrapolati da un dataset.
    \item L'identificazione di feature associate alle diagnosi.
    \item L'implementazione di un modello di apprendimento in grado di calcolare la probabilità che un tumore al seno sia benigno o maligno.
\end{itemize}

\subsection{Specifica PEAS}
\begin{itemize}
    \item Performance (misure di prestazione adottate per valutare l’operato di un agente), nel mio caso
    verranno valutate la precisione di classificazione e l’accuratezza.
    \item Environment (elementi che formano l’ambiente), nel mio caso è costituito dai dati clinici dei pazienti, inclusi parametri e misure relative al cancro al seno.
    \item Actuators (attuatori disponibili all’agente per intraprendere le azioni), nel mio caso sarà la capacità di predirre la presenza o assenza di cancro al seno.
    \item Sensors (sensori attraverso i quali l’agente riceve gli input percettivi), nel mio caso l’agente va ad acquisire dati clinici del paziente, tra cui risultati di esami, misure e caratteristiche correlate al cancro al seno.
\end{itemize}

\subsection{Caratteristiche dell'ambiente}